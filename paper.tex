\documentclass[conference]{IEEEtran}
\IEEEoverridecommandlockouts
% The preceding line is only needed to identify funding in the first footnote. If that is unneeded, please comment it out.
\usepackage{cite}
\usepackage{amsmath,amssymb,amsfonts}
\usepackage{algorithmic}
\usepackage{graphicx}
\usepackage{textcomp}
\usepackage{xcolor} 
\def\BibTeX{{\rm B\kern-.05em{\sc i\kern-.025em b}\kern-.08em
    T\kern-.1667em\lower.7ex\hbox{E}\kern-.125emX}}
\begin{document}

\title{Overview Of Programming Concepts In Robotics\\}

\author{
    \IEEEauthorblockN{
        Belana Roman
    }

    \IEEEauthorblockA{
        \textit{Institute of Flight Guidance} \\
        \textit{German Aerospace Center DLR}\\
        Brunswick, Germany \\
        Belana.Roman@dlr.de
    }
    \and
    \IEEEauthorblockN{
        Noah Wiederhold
        }
    \IEEEauthorblockA{  
        \textit{In­sti­tute of Flight Sys­tems} \\
        \textit{German Aerospace Center DLR}\\
        Brunswick, Germany \\
        Noah.Wiederhold@dlr.de
    }
}

\maketitle

\begin{abstract}
    This paper gives an overview of the programming concepts in robotics. It is intended to be used as a general source of information. The paper is structured in a way that the reader get to know a number of concepts one at a time. The paper is concluded with a summary of the programming concepts in robotics.
\end{abstract}

\begin{IEEEkeywords}
    robotics, programming, concepts, ai, augmented reality, virtual reality 
\end{IEEEkeywords}

\section{Introduction}
  
\section{Overview of Concepts}

old-fashioned concepts
widely used in the industry

 \begin{itemize}
    \item Online Concepts
        \begin{itemize}
            \item Playback
            \item Master-Slave
            \item Teach-in
            \item CAD/graphical based
        \end{itemize}
    \item Offline Concepts
        \begin{itemize}
            \item CAD/graphical based
            \item text based
            \item task based
            \item Simulation
        \end{itemize}
    \item Hybrid Concepts
 \end{itemize}    

new or theoretical concepts
not widely used in the industry, published in scientific papers

\begin{itemize}
    \item Semantic Robot Programming
    %\item AI
    %\item AR
%\item VR
\end{itemize}

\section{Online Concepts}

Programming with online concepts mean working with the active robot and its controls. %(TODO cite Grundlagen der Robotik Seite 186)
This concept is used to give a robot a new set of skills in a fast and easy way, where the programmer has the chance to observe the resulting behavior directly. %(TODO cite " Seite 187)
Commonly used concepts are Teach-in-Programming and Master-Slave-Programming.

%general information about online concepts

    \subsection{Teach-in}
    With Teach-in-Programming the programmer teaches the robot needed sequences of movements. Therefore the programmer moves the robot via control elements or buttons, so the system can save the needed movements parameters like position, joint coordinates or the state of grippers and "learn".  The movement of the robot can be controlled via consoles or so called "Teach Pendants", handheld programming devices. Usually, due to security, the movements are teached with decreased speed. Later on the program paramters like speed or accuracy can be adjusted to meet the needed specifications. Then the programm can be execute automatically, in which the robot moves through all stored positions one after the other and thus executes the planned sequence of movements. %(TODO cite " Seite 187-188)
    Usually there are three forms of movements are distinguished:
    \begin{itemize}
        \item Point-to-Point
        \item Continous Path
        \item Muli-Point
    \end{itemize}

    Play-back programming is for example a special from of Teach-in-Programming commonly used for Multi-Point. In this the robot is programmed by demonstrating the movement by touch or hand guidance with switched off actuators. Then the robot stores the positions of the joints and interpolates a smooth path with the given points, which can then be traversed as it was shown. %(TODO: cite " Seite 188-189)

    \subsection{Master-Slave}
    The Master-Slave-Concept gives the chance to program heavy robots via online programming wihtout having to move them manually. To do this, the programmer needs two coupled robots, a small one that is easy to move and the heavy robot whose capabilities are to be programmed. The programmer moves the small robot, the so called Master. These movements are then copied from the so called Slave, the heavy robot. Because of the need of two coupled robots, this programming concept is usually expensive and therefore only used for teleoperations, so for places humans can not visit easily like under water, irradiated areas or in space. %(TODO cite " Seite 190)

    \subsection{Disadvantages of online programming}
    Even though online programming makes it possible to specify motion sequences very precisely, this type of robot programming is not useful or even possible for all applications. This concept makes it impossible, for example, to control the program flow beyond the movements, to process sensor data or to perform mathematical calculations. In addition, online programming requires time, which is a great disadvantage within a manufacturing process. Within this time, the robot is withdrawn from the process or possibly the whole process has to be stopped for this time. For such problems, concepts of offline programming are used. %(TODO " Seite 190-191)

\section{Offline Concepts}

    Programming with offline concepts means programming the robot without the need to be in direct contact with an active robot. This concept is used to give a robot a new set of skills in an indirect way, without interrupting the overall production process.
    The finished program gets loaded onto the robot afterwards, resulting in a minimum of downtime of other processes. \cite[p. 186]{b4}

    \subsection{text based}

        The concept of text based programming is based on the idea of programming a robot by manually writing source code in the native language of the robot or a language which can be translated into the native ones. The source code containing a set of instructions gets compiled like in any other programming language and the resulting program gets loaded onto the robot. %cite this?
        This explicit programming approach is one of the oldest and most common programming concepts in robotics. %cite this?

        % what are problem solving languages?
        Most of the programming languages used in robotics are problem-solving languages, which means, that they contain special commands for certain tasks the robot can perform. A basic example would be a command for moving the robot from one point to another. This is commonly done by giving the coordinates of the initial point, a target point and arguments for how to interpolate between those points as parameter to the command. 
        The robot then calculates the path based on the parameters and moves accordingly. %cite this?
               
        Based on the 2019 market share of today's biggest robot manufacturers the most common languages used for text based programming are RAPID, KRL and KAREL. \cite{s1}

        For those languages many programming environments exist. Examples based on the mentioned would be RobotStudio from ABB, Officelite from KUKA and ROBOGUIDE from FANUC.

        In addition to the textual programming features most of the enviroments also offer a graphical programming interface.
        The main concepts used in these interfaces are controlling the robot by draggable points or with a simulated controller.
        The first method works by moving and rotating of for example the robot's arm on different axis. 
        The second approach is based on simulated controller similar to the ones used in teach-in programming of the robot.

        Some environments even make use of AR technologies to show and simulate the robot running the created program in the real world.
        
        %example robot studio %cite this

    \subsection{task based}

        Task based programming is a more abstract and implicit approach to programming a robot.
        Instead of describing what movement the robot should perform it defines the tasks the robot should perform.
       
        In comparison to text based programming it adds a layer of abstraction to the programming workflow. For example wouldn't you have to understand the complex physic when gripping an object any more, you could just use some kind of grip command which does the task for you. %cite

        Task based programming typically uses sensor information to add dynamic to the program while running. For example if the robot is supposed to pick up an object whose position is not precisely known, it would first check if the object is in the gripper and if not it would move to the object and pick it up. If the object is already in the gripper it would just move to the next task.
        %cite

        Again there are many different programming environments for task based programming. Some examples are RoboGuide from FANUC and Visual Components. RoboGuide defines tasks by using a drag and drop interface, these task can later be combined to form a program.
        Visual Components instead defines tasks by a flowchart.
        %cite

        The resulting task descriptions get translated into lines of code by a task transformer. These code lines then get processed like in the text based programming approach by compiling it. %cite this?

        

    \subsection{CAD/graphical based}

        uses a 3d scenery viewer which shows the simulated production environment

        can simulate multiple robots and multiple tasks apart from robot movements

        example visual components

        Some environments also offer to include a more mathematical approach to the programming workflow by adding different models and CAT based tools.
        Models could range from simple models of movements to more complex simulation models of the environment and the robot itself. %TODO cite this
        

    \subsection{Simulation}

\section{Hybrid Concepts}

    mixing of both offline and online concepts
    can be done by creating a general program offine and later  fine tune this programm with online concepts on the real robot
    or the other way around by first programming the robot online and then optimizing the program offline
    (b4 p 186)

\section{New Concepts} %?

    \subsection{Semantic Robot Programming}

        = programming by demonstrating

        in a paper ... the authors propose a new concept of programming robots by demonstrating the desired behavior to the robot by giving an initial state and a goal state for which the robot has to find the best way to get from the initial state to the goal state

        they use a new scene estimation method called DIGEST which splits the scene into a scene graph representing the scene structure of the initial state
        only requirement is the information about the number of objects present in the scene

        given now the structure of the initial state and the goal, a task planer to find the best way to get from the initial state to the goal state is used

        the task planer generates a series of actions to accomplish the task
        (p1 p 2)

    \subsection{AR}
    
        (p3)

    \subsection{VR}

        (p2)

% \section*{Acknowledgment}

% The preferred spelling of the word ``acknowledgment'' in America is without 
% an ``e'' after the ``g''. Avoid the stilted expression ``one of us (R. B. 
% G.) thanks $\ldots$''. Instead, try ``R. B. G. thanks$\ldots$''. Put sponsor 
% acknowledgments in the unnumbered footnote on the first page.

% \section*{References}

% Please number citations consecutively within brackets \cite{b1}. The 
% sentence punctuation follows the bracket \cite{b2}. Refer simply to the reference 
% number, as in \cite{b3}---do not use ``Ref. \cite{b3}'' or ``reference \cite{b3}'' except at 
% the beginning of a sentence: ``Reference \cite{b3} was the first $\ldots$''

% Number footnotes separately in superscripts. Place the actual footnote at 
% the bottom of the column in which it was cited. Do not put footnotes in the 
% abstract or reference list. Use letters for table footnotes.

% Unless there are six authors or more give all authors' names; do not use 
% ``et al.''. Papers that have not been published, even if they have been 
% submitted for publication, should be cited as ``unpublished'' \cite{b4}. Papers 
% that have been accepted for publication should be cited as ``in press'' \cite{b5}. 
% Capitalize only the first word in a paper title, except for proper nouns and 
% element symbols.

% For papers published in translation journals, please give the English 
% citation first, followed by the original foreign-language citation \cite{b6}.


\nocite{*}

\bibliographystyle{unsrt}
\bibliography{quellen}

\end{document}


%maybe add the statics cited in the text?