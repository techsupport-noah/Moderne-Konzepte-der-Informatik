\documentclass[conference]{IEEEtran}
\IEEEoverridecommandlockouts
% The preceding line is only needed to identify funding in the first footnote. If that is unneeded, please comment it out.
\usepackage{cite}
\usepackage{amsmath,amssymb,amsfonts}
\usepackage{algorithmic}
\usepackage{graphicx}
\usepackage{textcomp}
\usepackage{xcolor} 
\def\BibTeX{{\rm B\kern-.05em{\sc i\kern-.025em b}\kern-.08em
    T\kern-.1667em\lower.7ex\hbox{E}\kern-.125emX}}
\begin{document}

\title{Overview Of Programming Concepts In Robotics\\}

\author{
    \IEEEauthorblockN{
        Belana Roman
    }

    \IEEEauthorblockA{
        \textit{Institute of Flight Guidance} \\
        \textit{German Aerospace Center DLR}\\
        Brunswick, Germany \\
        Belana.Roman@dlr.de
    }
    \and
    \IEEEauthorblockN{
        Noah Wiederhold
        }
    \IEEEauthorblockA{  
        \textit{In­sti­tute of Flight Sys­tems} \\
        \textit{German Aerospace Center DLR}\\
        Brunswick, Germany \\
        Noah.Wiederhold@dlr.de
    }
}

\maketitle

\begin{abstract}
    This paper gives an overview of the programming concepts in robotics. It is intended to be used as a general source of information. The paper is structured in a way that the reader get to know a number of concepts one at a time. The paper is concluded with a summary of the programming concepts in robotics.
\end{abstract}

\begin{IEEEkeywords}
    robotics, programming, concepts, ai, augmented reality, virtual reality 
\end{IEEEkeywords}

\section{Introduction}
  
\section{Overview of Concepts}

old-fashioned concepts
widely used in the industry

 \begin{itemize}
    \item Online Concepts
        \begin{itemize}
            \item Playback
            \item Master-Slave
            \item Teach-in
            \item CAD/graphical based
        \end{itemize}
    \item Offline Concepts
        \begin{itemize}
            \item CAD/graphical based
            \item text based
            \item task based
            \item Simulation
        \end{itemize}
    \item Hybrid Concepts
 \end{itemize}    

new or theoretical concepts
not widely used in the industry, published in scientific papers

\begin{itemize}
    \item Semantic Robot Programming
    %\item AI
    %\item AR
%\item VR
\end{itemize}

\section{Online Concepts}

Programming with online concepts mean working with the active robot and its controls. %(TODO cite Grundlagen der Robotik Seite 186)
This concept is used to give a robot a new set of skills in a fast and easy way, where the programmer has the chance to observe the resulting behavior directly. %(TODO cite " Seite 187)
Commonly used concepts are Teach-in-Programming and Master-Slave-Programming.

%general information about online concepts

    \subsection{Teach-in}
    With Teach-in-Programming the programmer teaches the robot needed sequences of movements. Therefore the programmer moves the robot via control elements or buttons, so the system can save the needed movements parameters like position, joint coordinates or the state of grippers and "learn".  The movement of the robot can be controlled via consoles or so called "Teach Pendants", handheld programming devices. Usually, due to security, the movements are teached with decreased speed. Later on the program paramters like speed or accuracy can be adjusted to meet the needed specifications. Then the programm can be execute automatically, in which the robot moves through all stored positions one after the other and thus executes the planned sequence of movements. %(TODO cite " Seite 187-188)
    Usually there are three forms of movements are distinguished:
    \begin{itemize}
        \item Point-to-Point
        \item Continous Path
        \item Muli-Point
    \end{itemize}

    Play-back programming is for example a special from of Teach-in-Programming commonly used for Multi-Point. In this the robot is programmed by demonstrating the movement by touch or hand guidance with switched off actuators. Then the robot stores the positions of the joints and interpolates a smooth path with the given points, which can then be traversed as it was shown. %(TODO: cite " Seite 188-189)

    \subsection{Master-Slave}
    The Master-Slave-Concept gives the chance to program heavy robots via online programming wihtout having to move them manually. To do this, the programmer needs two coupled robots, a small one that is easy to move and the heavy robot whose capabilities are to be programmed. The programmer moves the small robot, the so called Master. These movements are then copied from the so called Slave, the heavy robot. Because of the need of two coupled robots, this programming concept is usually expensive and therefore only used for teleoperations, so for places humans can not visit easily like under water, irradiated areas or in space. %(TODO cite " Seite 190)

    \subsection{Disadvantages of online programming}
    Even though online programming makes it possible to specify motion sequences very precisely, this type of robot programming is not useful or even possible for all applications. This concept makes it impossible, for example, to control the program flow beyond the movements, to process sensor data or to perform mathematical calculations. In addition, online programming requires time, which is a great disadvantage within a manufacturing process. Within this time, the robot is withdrawn from the process or possibly the whole process has to be stopped for this time. For such problems, concepts of offline programming are used. %(TODO " Seite 190-191)

\section{Offline Concepts}

    development doesn't take place on an active robot itsself but on a seperate system, indirect programming
    the program gets loaded onto the robot later on

     (b4 p 186)

    \subsection{CAD/graphical based}
    \subsection{text based}

        text based = explicit
        you tell how to move to accomplish a certain task

        uses problem solving programming languages 
        gives access to commands for movements with specific parameters for the specific movement

        commands mostly define movements from on point to another and how to interpolate between them %source?
        
        programming enviroments for text based from the most common robot manufacturers %quote this?
        
        examples 
        abb robot studio
        uses rapid language %source?

        kuka officelite
        uses krl language %source?

        there are many programming languages to choose from developing using the textual concept
        based on the 2019 market share of todays biggest robot manufacturers the most commong languages used are KAREL, RAPID, KRL
        (s1)

        in addition to the textual programming features most of the enviroments also offer a graphical programming interface

        the main concepts used in these enviroments are controlling the robot by moving its arms on different axis and rotating it on those or to simulate a controller (representation of the ones used in the online concepts)

        kuka mainly uses the second concept where as abb uses both

        (https://new.abb.com/products/robotics/robotstudio)

        some enviroments even make use of ar technologie to show the roboter and its movements in the real world
        
        example robot studio

         

    
    \subsection{task based}

        task based = implicit
        you tell what task to do to achieve a certain goal
        more abstract than text based, adds a layer of abstraction to the programming workflow

        description of the tasks gets translated into lines of code by a task transformer (b1 p 116)

        uses sensor information to add dynamic to the program while running

        example task based programming enviroment:
        RoboGuide from FANUC defining by drag and drop
        Visual Components by flowchart
        Robotmaster 

    \subsection{CAD/graphical based}

        uses a 3d scenery viewer which shows the simulated production enviroment

        can simulate multiple robots and multiple tasks apart from robot movements

        example visual components

        

    \subsection{Simulation}

\section{Hybrid Concepts}

    mixing of both offline and online concepts
    can be done by creating a general program offine and later  fine tune this programm with online concepts on the real robot
    or the other way around by first programming the robot online and then optimizing the program offline
    (b4 p 186)

\section{New Concepts} %?

    \subsection{Semantic Robot Programming}

        = programming by demonstrating

        in a paper ... the authors propose a new concept of programming robots by demonstrating the desired behavior to the robot by giving an initial state and a goal state for which the robot has to find the best way to get from the initial state to the goal state

        they use a new scene estimation method called DIGEST which splits the scene into a scene graph representing the scene structure of the initial state
        only requirement is the information about the number of objects present in the scene

        given now the structure of the initial state and the goal, a task planer to find the best way to get from the initial state to the goal state is used

        the task planer generates a series of actions to accomplish the task
        (p1 p 2)

    \subsection{AR}
    
        (p3)

    \subsection{VR}

        (p2)

% \section*{Acknowledgment}

% The preferred spelling of the word ``acknowledgment'' in America is without 
% an ``e'' after the ``g''. Avoid the stilted expression ``one of us (R. B. 
% G.) thanks $\ldots$''. Instead, try ``R. B. G. thanks$\ldots$''. Put sponsor 
% acknowledgments in the unnumbered footnote on the first page.

% \section*{References}

% Please number citations consecutively within brackets \cite{b1}. The 
% sentence punctuation follows the bracket \cite{b2}. Refer simply to the reference 
% number, as in \cite{b3}---do not use ``Ref. \cite{b3}'' or ``reference \cite{b3}'' except at 
% the beginning of a sentence: ``Reference \cite{b3} was the first $\ldots$''

% Number footnotes separately in superscripts. Place the actual footnote at 
% the bottom of the column in which it was cited. Do not put footnotes in the 
% abstract or reference list. Use letters for table footnotes.

% Unless there are six authors or more give all authors' names; do not use 
% ``et al.''. Papers that have not been published, even if they have been 
% submitted for publication, should be cited as ``unpublished'' \cite{b4}. Papers 
% that have been accepted for publication should be cited as ``in press'' \cite{b5}. 
% Capitalize only the first word in a paper title, except for proper nouns and 
% element symbols.

% For papers published in translation journals, please give the English 
% citation first, followed by the original foreign-language citation \cite{b6}.


\nocite{*}

\bibliographystyle{unsrt}
\bibliography{quellen}

\end{document}


%maybe add the statics cited in the text?