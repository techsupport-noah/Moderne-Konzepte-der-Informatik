\documentclass[conference]{IEEEtran}
\IEEEoverridecommandlockouts
% The preceding line is only needed to identify funding in the first footnote. If that is unneeded, please comment it out.
\usepackage{cite}
\usepackage{amsmath,amssymb,amsfonts}
\usepackage{algorithmic}
\usepackage{graphicx}
\usepackage{textcomp}
\usepackage{xcolor} 
\def\BibTeX{{\rm B\kern-.05em{\sc i\kern-.025em b}\kern-.08em
    T\kern-.1667em\lower.7ex\hbox{E}\kern-.125emX}}
\begin{document}

\title{Overview Of Programming Concepts In Robotics\\}

\author{
    \IEEEauthorblockN{
        Belana Roman
    }

    \IEEEauthorblockA{
        \textit{Institute of Flight Guidance} \\
        \textit{German Aerospace Center DLR}\\
        Brunswick, Germany \\
        Belana.Roman@dlr.de
    }
    \and
    \IEEEauthorblockN{
        Noah Wiederhold
        }
    \IEEEauthorblockA{  
        \textit{In­sti­tute of Flight Sys­tems} \\
        \textit{German Aerospace Center DLR}\\
        Brunswick, Germany \\
        Noah.Wiederhold@dlr.de
    }
}

\maketitle

\begin{abstract}
    This paper gives an overview of the programming concepts in robotics. It is intended to be used as a general source of information. The paper is structured in a way that the reader get to know a number of concepts one at a time. The paper is concluded with a summary of the programming concepts in robotics.
\end{abstract}

\begin{IEEEkeywords}
    robotics, programming, concepts, ai, augmented reality, virtual reality
\end{IEEEkeywords}

\section{Introduction}
  
\section{Overview of Concepts}

 \begin{itemize}
    \item Online Concepts
        \begin{itemize}
            \item Playback
            \item Master-Slave
            \item Teach-in
            \item CAD/graphical based
        \end{itemize}
    \item Offline Concepts
        \begin{itemize}
            \item CAD/graphical based
            \item text based
            \item task based
                \begin{itemize}
                    \item explicit
                    \item implicit
                \end{itemize}
            \item Simulation
        \end{itemize}
    \item Hybrid Concepts
 \end{itemize}    

\section{Online Concepts}

Programming with online concepts mean working with the active robot and its controls. %(TODO cite Grundlagen der Robotik Seite 186)
This concept is used to give a robot a new set of skills in a fast and easy way, where the programmer has the chance to observe the resulting behavior directly. %(TODO cite " Seite 187)
Commonly used concepts are Teach-in-Programming and Master-Slave-Programming.

%general information about online concepts

    \subsection{Teach-in}
    With Teach-in-Programming the programmer teaches the robot needed sequences of movements. Therefore the programmer moves the robot via control elements or buttons, so the system can save the needed movements parameters like position, joint coordinates or the state of grippers and "learn".  The movement of the robot can be controlled via consoles or so called "Teach Pendants", handheld programming devices. Usually, due to security, the movements are teached with decreased speed. Later on the program paramters like speed or accuracy can be adjusted to meet the needed specifications. Then the programm can be execute automatically, in which the robot moves through all stored positions one after the other and thus executes the planned sequence of movements. %(TODO cite " Seite 187-188)
    Usually there are three forms of movements are distinguished:
    \begin{itemize}
        \item Point-to-Point
        \item Continous Path
        \item Muli-Point
    \end{itemize}

    Play-back programming is a from of Teach-in-Programming commonly used for Multi-Point. In this the robot is programmed by demonstrating the movement by touch or hand guidance with switched off actuators. Then the robot stores the positions of the joints and interpolates a smooth path with the given points, which can then be traversed as it was shown. %(TODO: cite " Seite 188-189)

    \subsection{Master-Slave}
    

\section{Offline Concepts}
    \subsection{CAD/graphical based}
    \subsection{text based}

        uses problem solving programming languages 
        gives access to commands for movements with specific parameters for the specific movement
        
        programming enviroments for text based programming:
            \begin{itemize}
                \item KUKA 
                \item ABB RobotStudio
                \item Fanuc RobotStudio
                \item Staubli TX
                \item Adept
                \item Motoman
            \end{itemize}
        (b1 p. 113)

    
    \subsection{task based}
        \subsubsection{explicit}
        \subsubsection{implicit}
    \subsection{CAD/graphical based}

    \subsection{Simulation}

\section{Hybrid Concepts}

\section*{Acknowledgment}

The preferred spelling of the word ``acknowledgment'' in America is without 
an ``e'' after the ``g''. Avoid the stilted expression ``one of us (R. B. 
G.) thanks $\ldots$''. Instead, try ``R. B. G. thanks$\ldots$''. Put sponsor 
acknowledgments in the unnumbered footnote on the first page.

\section*{References}

Please number citations consecutively within brackets \cite{b1}. The 
sentence punctuation follows the bracket \cite{b2}. Refer simply to the reference 
number, as in \cite{b3}---do not use ``Ref. \cite{b3}'' or ``reference \cite{b3}'' except at 
the beginning of a sentence: ``Reference \cite{b3} was the first $\ldots$''

Number footnotes separately in superscripts. Place the actual footnote at 
the bottom of the column in which it was cited. Do not put footnotes in the 
abstract or reference list. Use letters for table footnotes.

Unless there are six authors or more give all authors' names; do not use 
``et al.''. Papers that have not been published, even if they have been 
submitted for publication, should be cited as ``unpublished'' \cite{b4}. Papers 
that have been accepted for publication should be cited as ``in press'' \cite{b5}. 
Capitalize only the first word in a paper title, except for proper nouns and 
element symbols.

For papers published in translation journals, please give the English 
citation first, followed by the original foreign-language citation \cite{b6}.

\begin{thebibliography}{00}
    \bibitem{b1} Wolfgang Weber, Heiko Koch, ''Industrieroboter Methoden der Steuerung und Regelung'' Carl Hanser Verlag München, 5. Auflage, 2022

\end{thebibliography} 
\vspace{12pt}
\color{red}
IEEE conference templates contain guidance text for composing and formatting conference papers. Please ensure that all template text is removed from your conference paper prior to submission to the conference. Failure to remove the template text from your paper may result in your paper not being published.

\end{document}